% Options for packages loaded elsewhere
\PassOptionsToPackage{unicode}{hyperref}
\PassOptionsToPackage{hyphens}{url}
%
\documentclass[
]{article}
\usepackage{amsmath,amssymb}
\usepackage{lmodern}
\usepackage{iftex}
\ifPDFTeX
  \usepackage[T1]{fontenc}
  \usepackage[utf8]{inputenc}
  \usepackage{textcomp} % provide euro and other symbols
\else % if luatex or xetex
  \usepackage{unicode-math}
  \defaultfontfeatures{Scale=MatchLowercase}
  \defaultfontfeatures[\rmfamily]{Ligatures=TeX,Scale=1}
\fi
% Use upquote if available, for straight quotes in verbatim environments
\IfFileExists{upquote.sty}{\usepackage{upquote}}{}
\IfFileExists{microtype.sty}{% use microtype if available
  \usepackage[]{microtype}
  \UseMicrotypeSet[protrusion]{basicmath} % disable protrusion for tt fonts
}{}
\makeatletter
\@ifundefined{KOMAClassName}{% if non-KOMA class
  \IfFileExists{parskip.sty}{%
    \usepackage{parskip}
  }{% else
    \setlength{\parindent}{0pt}
    \setlength{\parskip}{6pt plus 2pt minus 1pt}}
}{% if KOMA class
  \KOMAoptions{parskip=half}}
\makeatother
\usepackage{xcolor}
\usepackage[margin=1in]{geometry}
\usepackage{color}
\usepackage{fancyvrb}
\newcommand{\VerbBar}{|}
\newcommand{\VERB}{\Verb[commandchars=\\\{\}]}
\DefineVerbatimEnvironment{Highlighting}{Verbatim}{commandchars=\\\{\}}
% Add ',fontsize=\small' for more characters per line
\usepackage{framed}
\definecolor{shadecolor}{RGB}{248,248,248}
\newenvironment{Shaded}{\begin{snugshade}}{\end{snugshade}}
\newcommand{\AlertTok}[1]{\textcolor[rgb]{0.94,0.16,0.16}{#1}}
\newcommand{\AnnotationTok}[1]{\textcolor[rgb]{0.56,0.35,0.01}{\textbf{\textit{#1}}}}
\newcommand{\AttributeTok}[1]{\textcolor[rgb]{0.77,0.63,0.00}{#1}}
\newcommand{\BaseNTok}[1]{\textcolor[rgb]{0.00,0.00,0.81}{#1}}
\newcommand{\BuiltInTok}[1]{#1}
\newcommand{\CharTok}[1]{\textcolor[rgb]{0.31,0.60,0.02}{#1}}
\newcommand{\CommentTok}[1]{\textcolor[rgb]{0.56,0.35,0.01}{\textit{#1}}}
\newcommand{\CommentVarTok}[1]{\textcolor[rgb]{0.56,0.35,0.01}{\textbf{\textit{#1}}}}
\newcommand{\ConstantTok}[1]{\textcolor[rgb]{0.00,0.00,0.00}{#1}}
\newcommand{\ControlFlowTok}[1]{\textcolor[rgb]{0.13,0.29,0.53}{\textbf{#1}}}
\newcommand{\DataTypeTok}[1]{\textcolor[rgb]{0.13,0.29,0.53}{#1}}
\newcommand{\DecValTok}[1]{\textcolor[rgb]{0.00,0.00,0.81}{#1}}
\newcommand{\DocumentationTok}[1]{\textcolor[rgb]{0.56,0.35,0.01}{\textbf{\textit{#1}}}}
\newcommand{\ErrorTok}[1]{\textcolor[rgb]{0.64,0.00,0.00}{\textbf{#1}}}
\newcommand{\ExtensionTok}[1]{#1}
\newcommand{\FloatTok}[1]{\textcolor[rgb]{0.00,0.00,0.81}{#1}}
\newcommand{\FunctionTok}[1]{\textcolor[rgb]{0.00,0.00,0.00}{#1}}
\newcommand{\ImportTok}[1]{#1}
\newcommand{\InformationTok}[1]{\textcolor[rgb]{0.56,0.35,0.01}{\textbf{\textit{#1}}}}
\newcommand{\KeywordTok}[1]{\textcolor[rgb]{0.13,0.29,0.53}{\textbf{#1}}}
\newcommand{\NormalTok}[1]{#1}
\newcommand{\OperatorTok}[1]{\textcolor[rgb]{0.81,0.36,0.00}{\textbf{#1}}}
\newcommand{\OtherTok}[1]{\textcolor[rgb]{0.56,0.35,0.01}{#1}}
\newcommand{\PreprocessorTok}[1]{\textcolor[rgb]{0.56,0.35,0.01}{\textit{#1}}}
\newcommand{\RegionMarkerTok}[1]{#1}
\newcommand{\SpecialCharTok}[1]{\textcolor[rgb]{0.00,0.00,0.00}{#1}}
\newcommand{\SpecialStringTok}[1]{\textcolor[rgb]{0.31,0.60,0.02}{#1}}
\newcommand{\StringTok}[1]{\textcolor[rgb]{0.31,0.60,0.02}{#1}}
\newcommand{\VariableTok}[1]{\textcolor[rgb]{0.00,0.00,0.00}{#1}}
\newcommand{\VerbatimStringTok}[1]{\textcolor[rgb]{0.31,0.60,0.02}{#1}}
\newcommand{\WarningTok}[1]{\textcolor[rgb]{0.56,0.35,0.01}{\textbf{\textit{#1}}}}
\usepackage{graphicx}
\makeatletter
\def\maxwidth{\ifdim\Gin@nat@width>\linewidth\linewidth\else\Gin@nat@width\fi}
\def\maxheight{\ifdim\Gin@nat@height>\textheight\textheight\else\Gin@nat@height\fi}
\makeatother
% Scale images if necessary, so that they will not overflow the page
% margins by default, and it is still possible to overwrite the defaults
% using explicit options in \includegraphics[width, height, ...]{}
\setkeys{Gin}{width=\maxwidth,height=\maxheight,keepaspectratio}
% Set default figure placement to htbp
\makeatletter
\def\fps@figure{htbp}
\makeatother
\setlength{\emergencystretch}{3em} % prevent overfull lines
\providecommand{\tightlist}{%
  \setlength{\itemsep}{0pt}\setlength{\parskip}{0pt}}
\setcounter{secnumdepth}{-\maxdimen} % remove section numbering
\ifLuaTeX
  \usepackage{selnolig}  % disable illegal ligatures
\fi
\IfFileExists{bookmark.sty}{\usepackage{bookmark}}{\usepackage{hyperref}}
\IfFileExists{xurl.sty}{\usepackage{xurl}}{} % add URL line breaks if available
\urlstyle{same} % disable monospaced font for URLs
\hypersetup{
  pdftitle={Script\_1.R},
  pdfauthor={jairo},
  hidelinks,
  pdfcreator={LaTeX via pandoc}}

\title{Script\_1.R}
\author{jairo}
\date{2023-02-09}

\begin{document}
\maketitle

\begin{Shaded}
\begin{Highlighting}[]
\CommentTok{\# Jairo Alberto Leal Gómez}
\CommentTok{\# 09/02/2023}
\CommentTok{\# Características descriptivas}


\CommentTok{\# Primera sesión {-}{-}{-}{-}{-}{-}{-}{-}{-}{-}{-}{-}{-}{-}{-}{-}{-}{-}{-}{-}{-}{-}{-}{-}{-}{-}{-}{-}{-}{-}{-}{-}{-}{-}{-}{-}{-}{-}{-}{-}{-}{-}{-}{-}{-}{-}{-}{-}{-}{-}{-}{-}{-}{-}{-}{-}{-}{-}}

\NormalTok{dbh }\OtherTok{\textless{}{-}} \DecValTok{15}
\NormalTok{h }\OtherTok{\textless{}{-}} \DecValTok{8}

\CommentTok{\# Multiplicación y operaciones}

\NormalTok{dbh }\SpecialCharTok{*}\NormalTok{ h}
\end{Highlighting}
\end{Shaded}

\begin{verbatim}
## [1] 120
\end{verbatim}

\begin{Shaded}
\begin{Highlighting}[]
\NormalTok{dbh}\SpecialCharTok{\^{}}\DecValTok{2}
\end{Highlighting}
\end{Shaded}

\begin{verbatim}
## [1] 225
\end{verbatim}

\begin{Shaded}
\begin{Highlighting}[]
\FunctionTok{log}\NormalTok{(dbh)}
\end{Highlighting}
\end{Shaded}

\begin{verbatim}
## [1] 2.70805
\end{verbatim}

\begin{Shaded}
\begin{Highlighting}[]
\NormalTok{dbh\_2 }\OtherTok{\textless{}{-}} \FunctionTok{c}\NormalTok{(}\DecValTok{12}\NormalTok{, }\DecValTok{8}\NormalTok{, }\DecValTok{7}\NormalTok{, }\DecValTok{5}\NormalTok{, }\DecValTok{11}\NormalTok{, }\DecValTok{13}\NormalTok{, }\DecValTok{16}\NormalTok{, }\DecValTok{21}\NormalTok{, }\DecValTok{8}\NormalTok{, }\DecValTok{16}\NormalTok{) }

\NormalTok{dbh\_2 }\SpecialCharTok{*}\NormalTok{ h}
\end{Highlighting}
\end{Shaded}

\begin{verbatim}
##  [1]  96  64  56  40  88 104 128 168  64 128
\end{verbatim}

\begin{Shaded}
\begin{Highlighting}[]
\NormalTok{h\_2 }\OtherTok{\textless{}{-}} \FunctionTok{c}\NormalTok{(}\DecValTok{5}\NormalTok{, }\DecValTok{3}\NormalTok{, }\FloatTok{2.4}\NormalTok{, }\DecValTok{3}\NormalTok{, }\FloatTok{4.7}\NormalTok{, }\FloatTok{5.8}\NormalTok{, }\DecValTok{7}\NormalTok{, }\DecValTok{11}\NormalTok{, }\FloatTok{2.4}\NormalTok{, }\FloatTok{7.2}\NormalTok{)}

\NormalTok{dbh\_2 }\SpecialCharTok{*}\NormalTok{ h\_2}
\end{Highlighting}
\end{Shaded}

\begin{verbatim}
##  [1]  60.0  24.0  16.8  15.0  51.7  75.4 112.0 231.0  19.2 115.2
\end{verbatim}

\begin{Shaded}
\begin{Highlighting}[]
\CommentTok{\# Graficas {-}{-}{-}{-}{-}{-}{-}{-}{-}{-}{-}{-}{-}{-}{-}{-}{-}{-}{-}{-}{-}{-}{-}{-}{-}{-}{-}{-}{-}{-}{-}{-}{-}{-}{-}{-}{-}{-}{-}{-}{-}{-}{-}{-}{-}{-}{-}{-}{-}{-}{-}{-}{-}{-}{-}{-}{-}{-}{-}{-}{-}{-}{-}{-}}

\CommentTok{\# Medidas de tendencia central}
\CommentTok{\# Media, moda}

\CommentTok{\# Medidas de dispersión}
\CommentTok{\# Varianza, desviación estandar, etc.}

\CommentTok{\# Representación gráfica de los datos}
\CommentTok{\# Histogramas, distribución de un solo grupo en un rango}
\CommentTok{\# Gráfica de barras, grupos }
\CommentTok{\# Dispersión}
\CommentTok{\# Boxplot, gráfica de cajas}

\FunctionTok{boxplot}\NormalTok{(dbh\_2, }\AttributeTok{col=} \StringTok{\textquotesingle{}blue\textquotesingle{}}\NormalTok{, }\AttributeTok{main =} \StringTok{\textquotesingle{}Boxplot DBH\textquotesingle{}}\NormalTok{)}
\end{Highlighting}
\end{Shaded}

\includegraphics{Script_1_files/figure-latex/unnamed-chunk-1-1.pdf}

\begin{Shaded}
\begin{Highlighting}[]
\CommentTok{\# Limites mínimo y máximo, centro representa la mediana, 50\% de los datos en la caja, y cada uno es un cuartil (25, 50, 75, 100)}

\FunctionTok{boxplot}\NormalTok{(h\_2, }\AttributeTok{col=} \StringTok{\textquotesingle{}green\textquotesingle{}}\NormalTok{, }\AttributeTok{main =} \StringTok{\textquotesingle{}Boxplot h\textquotesingle{}}\NormalTok{)}
\end{Highlighting}
\end{Shaded}

\includegraphics{Script_1_files/figure-latex/unnamed-chunk-1-2.pdf}

\begin{Shaded}
\begin{Highlighting}[]
\FunctionTok{plot}\NormalTok{(dbh\_2, h\_2, }\AttributeTok{col=} \StringTok{"red"}\NormalTok{)}
\end{Highlighting}
\end{Shaded}

\includegraphics{Script_1_files/figure-latex/unnamed-chunk-1-3.pdf}

\begin{Shaded}
\begin{Highlighting}[]
\CommentTok{\# Correlación: asociación positiva}

\FunctionTok{hist}\NormalTok{(dbh\_2)}
\end{Highlighting}
\end{Shaded}

\includegraphics{Script_1_files/figure-latex/unnamed-chunk-1-4.pdf}

\begin{Shaded}
\begin{Highlighting}[]
\FunctionTok{hist}\NormalTok{(h\_2)}
\end{Highlighting}
\end{Shaded}

\includegraphics{Script_1_files/figure-latex/unnamed-chunk-1-5.pdf}

\begin{Shaded}
\begin{Highlighting}[]
\CommentTok{\# Datos aleatorios}
\CommentTok{\# para que me genere datos random}


\CommentTok{\# algorito para que siempre me genere los mismos números, el numero en (  ) puede ser cualquiera}
\FunctionTok{set.seed}\NormalTok{(}\DecValTok{13}\NormalTok{)}
\NormalTok{obs}\FloatTok{.50} \OtherTok{\textless{}{-}} \FunctionTok{rnorm}\NormalTok{(}\DecValTok{50}\NormalTok{, }\AttributeTok{mean=} \DecValTok{3}\NormalTok{)}
\FunctionTok{hist}\NormalTok{(obs}\FloatTok{.50}\NormalTok{)}
\end{Highlighting}
\end{Shaded}

\includegraphics{Script_1_files/figure-latex/unnamed-chunk-1-6.pdf}

\begin{Shaded}
\begin{Highlighting}[]
\FunctionTok{set.seed}\NormalTok{(}\DecValTok{13}\NormalTok{)}
\NormalTok{if}\FloatTok{.50} \OtherTok{\textless{}{-}} \FunctionTok{runif}\NormalTok{(}\DecValTok{50}\NormalTok{, }\AttributeTok{min =} \DecValTok{10}\NormalTok{, }\AttributeTok{max =} \DecValTok{40}\NormalTok{)}
\FunctionTok{hist}\NormalTok{(if}\FloatTok{.50}\NormalTok{)}
\end{Highlighting}
\end{Shaded}

\includegraphics{Script_1_files/figure-latex/unnamed-chunk-1-7.pdf}

\begin{Shaded}
\begin{Highlighting}[]
\FunctionTok{set.seed}\NormalTok{(}\DecValTok{13}\NormalTok{)}
\NormalTok{if}\FloatTok{.100} \OtherTok{\textless{}{-}} \FunctionTok{runif}\NormalTok{(}\DecValTok{100}\NormalTok{, }\AttributeTok{min =} \DecValTok{10}\NormalTok{, }\AttributeTok{max =} \DecValTok{40}\NormalTok{)}
\FunctionTok{hist}\NormalTok{(if}\FloatTok{.100}\NormalTok{)}
\end{Highlighting}
\end{Shaded}

\includegraphics{Script_1_files/figure-latex/unnamed-chunk-1-8.pdf}

\begin{Shaded}
\begin{Highlighting}[]
\NormalTok{if}\FloatTok{.500} \OtherTok{\textless{}{-}} \FunctionTok{runif}\NormalTok{(}\DecValTok{500}\NormalTok{, }\AttributeTok{min =} \DecValTok{10}\NormalTok{, }\AttributeTok{max =} \DecValTok{40}\NormalTok{)}
\FunctionTok{hist}\NormalTok{(if}\FloatTok{.500}\NormalTok{)}
\end{Highlighting}
\end{Shaded}

\includegraphics{Script_1_files/figure-latex/unnamed-chunk-1-9.pdf}

\begin{Shaded}
\begin{Highlighting}[]
\NormalTok{if}\FloatTok{.1000} \OtherTok{\textless{}{-}} \FunctionTok{runif}\NormalTok{(}\DecValTok{1000}\NormalTok{, }\AttributeTok{min =} \DecValTok{10}\NormalTok{, }\AttributeTok{max =} \DecValTok{40}\NormalTok{)}
\FunctionTok{hist}\NormalTok{(if}\FloatTok{.1000}\NormalTok{)}
\end{Highlighting}
\end{Shaded}

\includegraphics{Script_1_files/figure-latex/unnamed-chunk-1-10.pdf}

\begin{Shaded}
\begin{Highlighting}[]
\FunctionTok{set.seed}\NormalTok{(}\DecValTok{1}\NormalTok{)}
\FunctionTok{stem}\NormalTok{(if}\FloatTok{.50}\NormalTok{)}
\end{Highlighting}
\end{Shaded}

\begin{verbatim}
## 
##   The decimal point is 1 digit(s) to the right of the |
## 
##   1 | 0112333344
##   1 | 77
##   2 | 001122344
##   2 | 666778888999
##   3 | 000012334
##   3 | 66678899
\end{verbatim}

\begin{Shaded}
\begin{Highlighting}[]
\FunctionTok{hist}\NormalTok{(if}\FloatTok{.50}\NormalTok{)}
\end{Highlighting}
\end{Shaded}

\includegraphics{Script_1_files/figure-latex/unnamed-chunk-1-11.pdf}

\begin{Shaded}
\begin{Highlighting}[]
\CommentTok{\# Restricciones {-}{-}{-}{-}{-}{-}{-}{-}{-}{-}{-}{-}{-}{-}{-}{-}{-}{-}{-}{-}{-}{-}{-}{-}{-}{-}{-}{-}{-}{-}{-}{-}{-}{-}{-}{-}{-}{-}{-}{-}{-}{-}{-}{-}{-}{-}{-}{-}{-}{-}{-}{-}{-}{-}{-}{-}{-}{-}{-}}

\CommentTok{\# Trabajar con datos del objeto if.50}

\FunctionTok{mean}\NormalTok{(if}\FloatTok{.50}\NormalTok{)}
\end{Highlighting}
\end{Shaded}

\begin{verbatim}
## [1] 25.3432
\end{verbatim}

\begin{Shaded}
\begin{Highlighting}[]
\FunctionTok{fivenum}\NormalTok{(if}\FloatTok{.50}\NormalTok{)}
\end{Highlighting}
\end{Shaded}

\begin{verbatim}
## [1] 10.32800 19.87381 27.10863 31.30967 39.00500
\end{verbatim}

\begin{Shaded}
\begin{Highlighting}[]
\FunctionTok{boxplot}\NormalTok{(if}\FloatTok{.50}\NormalTok{)}
\end{Highlighting}
\end{Shaded}

\includegraphics{Script_1_files/figure-latex/unnamed-chunk-1-12.pdf}

\begin{Shaded}
\begin{Highlighting}[]
\CommentTok{\# mean= media   ;   fivenum= 5 datos del boxplot, rangos, limites de cuartiles y mediana}

\CommentTok{\# SIMBOLOS DE RESTRICCIONES DE DATOS}

\CommentTok{\# igual a ==}
\CommentTok{\# diferente a !=}
\CommentTok{\# igual o mayor \textgreater{}=}
\CommentTok{\# igual o menor \textless{}=}
\CommentTok{\# mayor que \textgreater{}}
\CommentTok{\# menor que \textless{}}

\NormalTok{if}\FloatTok{.50} \SpecialCharTok{\textless{}=} \FunctionTok{median}\NormalTok{(if}\FloatTok{.50}\NormalTok{)}
\end{Highlighting}
\end{Shaded}

\begin{verbatim}
##  [1] FALSE  TRUE  TRUE  TRUE FALSE  TRUE FALSE FALSE FALSE  TRUE FALSE FALSE
## [13] FALSE  TRUE FALSE  TRUE  TRUE FALSE FALSE FALSE  TRUE  TRUE FALSE  TRUE
## [25]  TRUE FALSE  TRUE  TRUE  TRUE FALSE  TRUE  TRUE FALSE FALSE  TRUE  TRUE
## [37]  TRUE  TRUE FALSE  TRUE FALSE  TRUE FALSE FALSE FALSE FALSE  TRUE  TRUE
## [49] FALSE FALSE
\end{verbatim}

\begin{Shaded}
\begin{Highlighting}[]
\CommentTok{\# dice que datos cumplen la restricción... instruccion lógica}

\CommentTok{\# submuestreo dirigido}

\NormalTok{dbh}\FloatTok{.50} \OtherTok{\textless{}{-}} \FunctionTok{subset}\NormalTok{(if}\FloatTok{.50}\NormalTok{, if}\FloatTok{.50} \SpecialCharTok{\textless{}=} \FunctionTok{median}\NormalTok{(if}\FloatTok{.50}\NormalTok{))}

\NormalTok{dbh.up50 }\OtherTok{\textless{}{-}} \FunctionTok{subset}\NormalTok{(if}\FloatTok{.50}\NormalTok{, if}\FloatTok{.50} \SpecialCharTok{\textgreater{}=} \FunctionTok{median}\NormalTok{(if}\FloatTok{.50}\NormalTok{))}

\NormalTok{dbh.up30 }\OtherTok{\textless{}{-}} \FunctionTok{subset}\NormalTok{(if}\FloatTok{.50}\NormalTok{, if}\FloatTok{.50} \SpecialCharTok{\textgreater{}} \DecValTok{30}\NormalTok{)}
\NormalTok{dbh.up30}
\end{Highlighting}
\end{Shaded}

\begin{verbatim}
##  [1] 31.30967 38.86194 32.93194 36.20147 36.35113 36.71677 35.96354 30.41571
##  [9] 30.33739 37.56123 34.26281 32.76192 39.00500 37.56994 31.97295 30.41875
\end{verbatim}

\begin{Shaded}
\begin{Highlighting}[]
\FunctionTok{mean}\NormalTok{(dbh.up30)}
\end{Highlighting}
\end{Shaded}

\begin{verbatim}
## [1] 34.54013
\end{verbatim}

\begin{Shaded}
\begin{Highlighting}[]
\FunctionTok{sd}\NormalTok{(dbh.up30)}
\end{Highlighting}
\end{Shaded}

\begin{verbatim}
## [1] 3.100909
\end{verbatim}

\begin{Shaded}
\begin{Highlighting}[]
\CommentTok{\# Importar datos {-}{-}{-}{-}{-}{-}{-}{-}{-}{-}{-}{-}{-}{-}{-}{-}{-}{-}{-}{-}{-}{-}{-}{-}{-}{-}{-}{-}{-}{-}{-}{-}{-}{-}{-}{-}{-}{-}{-}{-}{-}{-}{-}{-}{-}{-}{-}{-}{-}{-}{-}{-}{-}{-}{-}{-}{-}{-}}

\NormalTok{fert }\OtherTok{\textless{}{-}} \FunctionTok{read.csv}\NormalTok{(}\StringTok{"vivero.csv"}\NormalTok{, }\AttributeTok{header =} \ConstantTok{TRUE}\NormalTok{)}

\CommentTok{\# mis datos no se exportaron como un factor, por lo cual no lo toma como grupos diferentes, siguiente comando para hacerlo}

\NormalTok{fert}\SpecialCharTok{$}\NormalTok{Tratamiento }\OtherTok{\textless{}{-}}\FunctionTok{as.factor}\NormalTok{(fert}\SpecialCharTok{$}\NormalTok{Tratamiento)}

\CommentTok{\# \textasciitilde{} en función de...}

\FunctionTok{boxplot}\NormalTok{(fert}\SpecialCharTok{$}\NormalTok{IE }\SpecialCharTok{\textasciitilde{}}\NormalTok{ fert}\SpecialCharTok{$}\NormalTok{Tratamiento, }
        \AttributeTok{main =} \StringTok{"Vivero Bosque Escuela"}\NormalTok{,}
        \AttributeTok{xlab =} \StringTok{"Tratamiento"}\NormalTok{, }\AttributeTok{ylab =} \StringTok{"Indice de Esbeltez"}\NormalTok{,}
        \AttributeTok{col =} \FunctionTok{c}\NormalTok{(}\StringTok{"red"}\NormalTok{, }\StringTok{"blue"}\NormalTok{),}
        \AttributeTok{las =} \DecValTok{1}\NormalTok{, }\AttributeTok{ylim =} \FunctionTok{c}\NormalTok{(}\FloatTok{0.4}\NormalTok{, }\FloatTok{1.2}\NormalTok{))}
\end{Highlighting}
\end{Shaded}

\includegraphics{Script_1_files/figure-latex/unnamed-chunk-1-13.pdf}

\end{document}
